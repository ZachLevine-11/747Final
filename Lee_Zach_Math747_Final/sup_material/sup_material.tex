%%%%%%%%%%%%%%%%%%%%%%%%%%%%%%%%%%%%%%%%%%%%%%%%%%%%%%%%%%%%%%%%%%%%%%%%%%%
% PHAC 2021 COVID-19 FORECAST REPORT (SUPPLEMENTARY MATERIAL)
% Analysts: Lee van Brussel & Zachary Levine
%%%%%%%%%%%%%%%%%%%%%%%%%%%%%%%%%%%%%%%%%%%%%%%%%%%%%%%%%%%%%%%%%%%%%%%%%%%
 
%--------------------------------------------------------------------------
%   LaTeX PACKAGES AND OTHER DOCUMENT CONFIGURATIONS
%--------------------------------------------------------------------------

\documentclass[12pt]{article}\usepackage[]{graphicx}\usepackage[]{color}
% maxwidth is the original width if it is less than linewidth
% otherwise use linewidth (to make sure the graphics do not exceed the margin)
\makeatletter
\def\maxwidth{ %
  \ifdim\Gin@nat@width>\linewidth
    \linewidth
  \else
    \Gin@nat@width
  \fi
}
\makeatother

\definecolor{fgcolor}{rgb}{0.345, 0.345, 0.345}
\newcommand{\hlnum}[1]{\textcolor[rgb]{0.686,0.059,0.569}{#1}}%
\newcommand{\hlstr}[1]{\textcolor[rgb]{0.192,0.494,0.8}{#1}}%
\newcommand{\hlcom}[1]{\textcolor[rgb]{0.678,0.584,0.686}{\textit{#1}}}%
\newcommand{\hlopt}[1]{\textcolor[rgb]{0,0,0}{#1}}%
\newcommand{\hlstd}[1]{\textcolor[rgb]{0.345,0.345,0.345}{#1}}%
\newcommand{\hlkwa}[1]{\textcolor[rgb]{0.161,0.373,0.58}{\textbf{#1}}}%
\newcommand{\hlkwb}[1]{\textcolor[rgb]{0.69,0.353,0.396}{#1}}%
\newcommand{\hlkwc}[1]{\textcolor[rgb]{0.333,0.667,0.333}{#1}}%
\newcommand{\hlkwd}[1]{\textcolor[rgb]{0.737,0.353,0.396}{\textbf{#1}}}%
\let\hlipl\hlkwb

\usepackage{framed}
\makeatletter
\newenvironment{kframe}{%
 \def\at@end@of@kframe{}%
 \ifinner\ifhmode%
  \def\at@end@of@kframe{\end{minipage}}%
  \begin{minipage}{\columnwidth}%
 \fi\fi%
 \def\FrameCommand##1{\hskip\@totalleftmargin \hskip-\fboxsep
 \colorbox{shadecolor}{##1}\hskip-\fboxsep
     % There is no \\@totalrightmargin, so:
     \hskip-\linewidth \hskip-\@totalleftmargin \hskip\columnwidth}%
 \MakeFramed {\advance\hsize-\width
   \@totalleftmargin\z@ \linewidth\hsize
   \@setminipage}}%
 {\par\unskip\endMakeFramed%
 \at@end@of@kframe}
\makeatother

\definecolor{shadecolor}{rgb}{.97, .97, .97}
\definecolor{messagecolor}{rgb}{0, 0, 0}
\definecolor{warningcolor}{rgb}{1, 0, 1}
\definecolor{errorcolor}{rgb}{1, 0, 0}
\newenvironment{knitrout}{}{} % an empty environment to be redefined in TeX

\usepackage{alltt}

\input{4mbapreamble.tex} % Document formatting

\usepackage{float} % This helps keep figures where you type them

\floatplacement{figure}{H} % Type "H" in figure environment (for float package)

\usepackage{booktabs} % Makes fancy tables

\usepackage{pdfpages} % Allows one to insert a PDF into the document
\IfFileExists{upquote.sty}{\usepackage{upquote}}{}
\begin{document}

%--------------------------------------------------------------------------
%   SUPPLEMENTARY MATERIAL FOR PHAC REPORT
%--------------------------------------------------------------------------

\section*{Supplementary Material}
The following pages contain supplementary material explaining calibration efforts, forecasting methods and additional assumptions for the submitted PHAC 2021 Forecast report.

%--------------------------------------------------------------------------
%   CALIBRATION METHODS
%--------------------------------------------------------------------------

\subsection*{Calibration Methods}

\subsubsection*{Setup for Calibration}

Before any forecasting can be done, it is important that critical parameters are calibrated to ensure forecasting accuracy. The calibration process begins by loading the observed data and manipulating it wherever necessary. Due to noise in the eastern provinces and low incidence in the territories, the Canadian data is filtered to contain only BC, AB, SK, MB, ON and QC. The removal of missing values and negative incidence is performed and then the data is converted to long form so that it can be passed to the \texttt{McMasterPandemic} function \texttt{calibrate}.




\subsubsection*{Original (Bad) Calibration Attempt}
Calibrations are done province-by-province. We began with setting the correct population size $N$ and optimizing for the initial number of exposed individuals $E_0$ and the initial transmission rate $\beta_0$. The calibration function was also fed an estimated reproduction number $\R_0$ based on the second wave data using the \Rlogo package \texttt{epigrowthfit}. It is clear that after calibration, this is not enough to accurately match the observed data. 





%% Additional commenting on the bad fits.

\subsubsection*{Updated Calibration Method Used for PHAC Report}

One of the main issues with the original parameter calibration is that it does not take into consideration the noise in the observed data. To rectify this, it was found that the function \texttt{calibrate} can \url{https://github.com/bbolker/McMasterPandemic/blob/master/ontario/Ontario_current.R}

\end{document}
